%%%%%%%%% PROJECT SUMMARY -- 1 page, third person
% e.g:  "The PI will prove" not "I will prove"

%Below are the pagination, font size, spacing and margin
%instructions for NSF proposals: \\
%
%FastLane does not automatically paginate a proposal.
%Each section of the proposal must be individually
%paginated prior to upload to the system. \\
%
%Use Computer Modern family of fonts at a font size of 11 points or
%larger. A font size of less than 10 points may be used for mathematical
%formulas or equations, figure, table or diagram captions and when
%using a Symbol font to insert Greek letters or special characters.
%The text must still be readable. The use of small type not in compliance with the NSF guidelines
%may be grounds for NSF to return the proposal without review. \\
%
%No more than 6 lines of text within a vertical space of 1 inch. \\
%
%Margins, in all directions, must be at least an inch. \\
%
%
\required{Project Summary}

\paragraph{Overview} The proposed project will investigate the genome-wide effects of hybridization and introgression in the genus \emph{Zea}.  First, investigators will study two independent hybrid zones that naturally occur in mid-elevations of Mexico between the lowland-adapted \emph{Z. mays} ssp. \emph{parviglumis} and the highland-adapted \emph{Z. mays} ssp. \emph{mexicana}.  Through field collections, generation of phenotypic data in common garden studies, and genotyping, the investigators will assess how fitness of these taxa varies across a hybrid zone, the level of convergence in the genetic architecture of introgression across independent hybrid populations and zones, and the evidence for selection on putatively adaptive phenotypic traits across an elevation gradient. Second, investigators will determine the impact of hybridization and introgression between domesticated maize (\emph{Z. mays} ssp. \emph{mays}) and wild  \emph{Zea} as maize spread from its center of origin. Population genomic analyses of sympatric collections will be used to test hypotheses about the significance of divergence time in patterning introgression between these taxa, the geographic scale of adaptive introgression, and the potential for maize to serve as a bridge for gene flow between otherwise allopatric and narrowly-distributed \emph{Zea} species.

% This should be a brief statement of the problem you plan to address.
% It should look something like an abstract. 

%The project summary should be a description of the proposed activity suitable
%for publication, no more than one page in length. It should not be
%an abstract of the proposal, but rather a self-contained description of
%the activity that would result if the proposal were funded. The summary
%should be written in the third person and include a statement of objectives
%and methods to be employed. It should be informative to other persons
%working in the same or related fields and understandable to a scientifically
%or technically literate lay reader. \\
%
%The summary must clearly address in separate statements (within the one-page summary):
%the intellectual merit of the proposed activity; and the broader impacts
%resulting from the proposed activity. Proposals that do not separately
%address both criteria within the one-page Project Summary will be returned without
%review. \\

\paragraph{Intellectual Merit}  Much progress has been made in the study of hybridization and introgression through the development of theory, through field-based ecological research, and through genetic analyses based on a limited number of molecular markers. However, much remains to be discovered regarding how these evolutionary processes have shaped genomes. The research proposed here will leverage the genomic resources of the maize model system to investigate how hybridization and introgression have molded the genomes of both wild \emph{Zea} species and domesticated maize by 1) Generating novel diversity that has potentially been adaptive in two independent hybrid zones of \emph{Zea mays} subspecies \emph{parviglumis} and \emph{mexicana}; and 2) Facilitating the spread of maize across the Americas through transference of local adaptation from local wild species to maize. These investigations will generate basic knowledge on the level of convergence in introgression in replicate hybrid zones and the role of introgression in facilitating rapid local adaptation during the spread of a colonizing species.

% This is why your project is interesting and will help further
% knowledge in the field of mathematics. 

%How important is the proposed activity to advancing
%knowledge and understanding within its own field or across different fields?
%How well qualified is the proposer (individual or team) to conduct the project?
%(If appropriate, the reviewer will comment on the quality of prior work.)
%To what extent does the proposed activity suggest and explore creative, original,
%or potentially transformative concepts? How well conceived and organized is the
%proposed activity? Is there sufficient access to resources?  \\

\paragraph{Broader Impacts}

The investigators will achieve societally relevant outcomes in the proposed project by providing STEM training opportunities for undergraduate and graduate students and establishing an exchange program between universities in the United States and Mexico.
The investigators have an excellent track record of providing training opportunities for both undergraduate and graduate students in laboratory, computational, and field-based research, have successfully recruited minority students into their research programs in the past and will make every effort to do so as part of this proposed work.
In addition, a teaching module on hybridization will be developed for the undergraduate core course "Principles of Biology" at Iowa State University based on the research proposed here.
This module will actively engage undergraduates in STEM-based research and will serve as a recruiting tool for independent research supported by this project in the investigators' laboratories.
Finally, a proposed exchange program will create an opportunity for students from the United States to conduct research internationally and allow these students to interact with visiting students from Mexico. Through these interactions, students will be better prepared for modern STEM research, which is often highly collaborative and international in nature.

% There are 4 kinds of broader impacts.
% 1. advance discovery and understanding while promoting teaching,
% training and learning
% 2. broaden the participation of underrepresented groups
% 3. disseminated broadly to enhance scientific and technological
% understanding
% 4. benefits of the proposed activity to society